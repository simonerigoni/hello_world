%%%%%%%%%%%%%%%%%%%%%%%%%%%%%%%%%%%%%%%%%
% Developer CV
% LaTeX Template
% Version 1.0 (28/1/19)
%
% This template originates from:
% http://www.LaTeXTemplates.com
%
% Authors:
% Jan Vorisek (jan@vorisek.me)
% Based on a template by Jan Küster (info@jankuester.com)
% Modified for LaTeX Templates by Vel (vel@LaTeXTemplates.com)
%
% License:
% The MIT License (see included LICENSE file)
%
%%%%%%%%%%%%%%%%%%%%%%%%%%%%%%%%%%%%%%%%%

%----------------------------------------------------------------------------------------
%	PACKAGES AND OTHER DOCUMENT CONFIGURATIONS
%----------------------------------------------------------------------------------------

\documentclass[9pt]{developercv} % Default font size, values from 8-12pt are recommended

%----------------------------------------------------------------------------------------

\begin{document}

%----------------------------------------------------------------------------------------
%	TITLE AND CONTACT INFORMATION
%----------------------------------------------------------------------------------------

\begin{minipage}[t]{0.375\textwidth} % 45% of the page width for name
	\vspace{-\baselineskip} % Required for vertically aligning minipages
	
	% If your name is very short, use just one of the lines below
	% If your name is very long, reduce the font size or make the minipage wider and reduce the others proportionately
	\colorbox{black}{{\HUGE\textcolor{white}{\textbf{\MakeUppercase{Simone}}}}} % First name
	
	\colorbox{black}{{\HUGE\textcolor{white}{\textbf{\MakeUppercase{Rigoni}}}}} % Last name
	
	\vspace{6pt}
	
	{\huge Technical Consultant} % Career or current job title
\end{minipage}
\begin{minipage}[t]{0.225\textwidth} % 27.5% of the page width for the first row of icons
	\vspace{-\baselineskip} % Required for vertically aligning minipages
	
	% The first parameter is the FontAwesome icon name, the second is the box size and the third is the text
	% Other icons can be found by referring to fontawesome.pdf (supplied with the template) and using the word after \fa in the command for the icon you want
	\includegraphics[width=3cm]{image_1}
\end{minipage}
\begin{minipage}[t]{0.300\textwidth} % 27.5% of the page width for the second row of icons
	\vspace{-\baselineskip} % Required for vertically aligning minipages
	
	% The first parameter is the FontAwesome icon name, the second is the box size and the third is the text
	% Other icons can be found by referring to fontawesome.pdf (supplied with the template) and using the word after \fa in the command for the icon you want
	\icon{MapMarker}{10}{Pavia, Lombardy, Italy}\\
	\icon{Phone}{10}{}\\
	\icon{At}{10}{\href{mailto:simone.rigoni01@gmail.com}{simone.rigoni01@gmail.com}}\\	
	\icon{Medium}{10}{\href{https://medium.com/@simone.rigoni01}{medium.com/@simone.rigoni01}}\\
	\icon{Github}{10}{\href{https://github.com/simonerigoni}{github.com/simonerigoni}}\\
	\icon{Linkedin}{10}{\href{https://www.linkedin.com/in/simone-rigoni-852b40101}{linkedin.com/in/simone-rigoni-852b40101}}\\
\end{minipage}

\vspace{0.5cm}

%----------------------------------------------------------------------------------------
%	INTRODUCTION, SKILLS AND TECHNOLOGIES
%----------------------------------------------------------------------------------------

\cvsect{Who Am I?}

\begin{minipage}[t]{0.5\textwidth} % 40% of the page width for the introduction text
	\vspace{-\baselineskip} % Required for vertically aligning minipages
	
	I graduated in Computer Engineering - Embedded and Control Systems at the University of Pavia in 2017. During my Erasmus Traineeship in Dublin I worked as intern at Firmwave, a young company specialized in the design of low-power hardware for IoT. I got passionate about Business Intelligence, Machine Learning and Data Science so I decided to completely change my working field. I started working in 2017 as Technical Consultant at Aptos which operates a singular commerce platform that enables retail enterprises to deliver omni-channel shopping experiences to customers. In 2019 I graduated from the Udacity Data Scientist Nanodegree Program. This course allowed me to deepen my theoretical and practical knowledge of Supervise, Unsupervised and Deep Learning
\end{minipage}
\hfill % Whitespace between
\begin{minipage}[t]{0.4\textwidth} % 50% of the page for the skills bar chart
	\vspace{-\baselineskip} % Required for vertically aligning minipages
	\begin{barchart}{5.5}
		\baritem{Python}{90}
		\baritem{T-SQL}{80}
		\baritem{MATLAB}{40}
		\baritem{HTML/CSS}{40}
		\baritem{PHP}{20}
		\baritem{Java}{20}
		\baritem{C++}{20}
		\baritem{C}{60}
		\baritem{MIPS}{40}
		\baritem{VBA}{40}
	\end{barchart}
\end{minipage}

\begin{center}
	\bubbles{5/Visual Studio, 4/Git, 4/Office, 5/SSMS, 3/Spark, 3/AD Inventor, 4/TFS, 4/LTSpice, 3/PowerBI, 3/Jira}
\end{center}

%----------------------------------------------------------------------------------------
%	EXPERIENCE
%----------------------------------------------------------------------------------------

\cvsect{Experience}

\begin{entrylist}
	\entry
		{10/2017 -- Currently}
		{Technical Consultant}
		{Aptos}
		{Mainly I develop T-SQL functionalities using Microsoft SQL Server, Microsoft SQL Server Analysis Services , Microsoft SQL Server Management Studio, Microsoft SQL Profiler, Microsoft Visual Studio and Team Foundation Server}
		{}
		{}
		%\texttt{T-SQL}\slashsep\texttt{VBA}\slashsep\texttt{MDX}\slashsep\texttt{MDX}}
	\entry
		{06/2017 -- 12/2017}
		{Waiter}
		{Corrado Calza Food \& Co., Milan}
		{Waiter for fashion related events}
		{}
		{}
	\entry
		{09/2015 -- 07/2017}
		{Tutor}
		{Plinio Fraccaro College}
		{Information Technology bases, SQL, Pytohn and C programming tutor for the students of Engineering and Mathematics}
		{}
		{}
	\entry
		{03/2016 -- 06/2016}
		{Software Developer}
		{Firmwave}
		{I took part in the design and implementation of Over the Air in Application Programming on a device based on STM32F4 MCU (equipped with ARM Cortex M4), implementing new features both in the bootloader and in the application, so that it could download updates from re-mote servers and reprogram. The code was written in C, with an open source toolchain based on GCC ARM + Eclipse + OpenOCD. I also took part in the development of Python based scripts for automated testing purposes}
		{}
		{}
	\entry
		{06/2009 -- 09/2009}
		{Intern}
		{Aglietta Mario - IT and office}
		{I worked with IT technicians to provide customer and technical support}
		{}
		{}
\end{entrylist}

%----------------------------------------------------------------------------------------
%	EDUCATION
%----------------------------------------------------------------------------------------

\cvsect{Education}

\begin{entrylist}
	\entry
		{10/2014 -- 10/2017}
		{Master's Degree (LM-32) in Computer Engineering - Embedded and Control Systems}
		{University of Pavia}
		{The degree program includes courses in electrical engineering, electrical drives and automation, advanced control systems, in-dustrial automation, process control, robotics, embedded systems and in real time.The lessons, held in English, are com-plemented by work laboratories, seminars and tutorials, held by industry leaders and professors from all over the world}
		{100/110}
		{Thesis: Design and implementation of a Bluetooth based indoor localization system for the sport domain}
	\entry
		{09/2011 -- 12/2014}
		{Bachelor’s Degree (L-8) in Computer Engineering}
		{University of Pavia}
		{The degree course is aimed at imparting lessons in various fields such as mathematics, physics, information technology, auto-mation, electromagnetism, electronics and telecommunications}
		{95/110}
		{Thesis: Stable stationary flight of a quadcopter}
	\entry
		{09/2007 -- 07/2011}
		{Diploma of Technical Institute Technological Sector - Informatics and Telecommunications}
		{Higher Education Institution Enrico Fermi of Mantova}
		{The course provides basic knowledge in computer science and electronics, in order to develop the skills necessary to analyze, size, manage and design small systems for processing, transmission, acquisition of the required information in symbolic form and in the form of signals electric}
		{73/110}
		{Thesis: On line management of an electronics store}
\end{entrylist}

%----------------------------------------------------------------------------------------
%	OTHER ACTIVITIES
%----------------------------------------------------------------------------------------

\cvsect{Other activities}

\begin{entrylist}
	\entry
		{01/2019 -- 10/2019}
		{Data Scientist Nanodegree}
		{Udacity}
		{The Data Scientist Nanodegree program is an advanced program designed to prepare the student for data scientist jobs. The topics coverd are Supervised Learning, Deep Learning, Unsupervised Learning, Data storytelling, Software development and Recommendation Engines. All the theoretical concepts explained during the lessons are then tested with quiz and projects that can be find on \underline{\href{https://github.com/simonerigoni/udacity/tree/master/data_scientist_nanodegree}{Github}}}
		{}
		{Capstone Project: User churn prediction using Spark}
	\entry
		{06/2019 -- Currently}
		{Content Writer}
		{Medium -  Towards Data Science}
		{I started writing for Medium through the Data Scientist Nanodegree and got passionate about it. I usually publish my stories on Towards Data Science which is one of the biggest platform on Medium for data science. All my stories can be find on \underline{\href{https://medium.com/@simone.rigoni01}{Medium}}}
		{}
		{}
	%\entry
	%	{09/2016 -- 07/2017}
	%	{Head of conferences}
	%	{Plinio Fraccaro College}
	%	{I dealt with the organization of conferences held at the Pavia College where I lived. I was one of the main promoters of one of the biggest %multidisciplinary conference ever organized in the college: "Artificial intelligence: techniques, applications and problems" aimed at highlighting %the salient features of artificial intelligence, also in disciplines such as philosophy, jurisprudence and computational linguistics. The professor %involved were from the University of Pavia and the IUSS Pavia}
	%	{}
	%	{}
	\entry
		{09/2018 -- 09/2018}
		{Publication: A Comparison of RSSI Filtering Techniques for Range-based Localization}
		{IEEE}
		{Received Signal Strength Indication (RSSI) is commonly used to provide distance estimates in range-based localization. In most cases, the localization systems use RSS at short range where the distance estimates are more reliable or use RSS alongside other techniques such as Time of Flight (ToF). More information can be find on \underline{\href{https://ieeexplore.ieee.org/abstract/document/8502556}{IEEE Explore}}}
		{}
		{}
	%\entry
	%	{03/2018 -- 03/2018}
	%	{Partecipant}
	%	{University2Business}
	%	{Smart Distribution Pack Management contest organized by University2Business in collaboration with TXT e-solutions. The contest required to %identify a method to optimize the process of retail distribution of textile products, with the application to a specific case. Together with a %colleague, we developed a MySQL database and a series of MATLAB scripts aimed at increasing the efficiency of the existing system by creating %pre-packaged packs containing products of various sizes to be sent to stores}
	%	{}
	%	{}
	%\entry
	%	{04/2016 -- 04/2016}
	%	{Partecipant}
	%	{Dublin Airport Hackathon}
	%	{The event was aimed at finding innovative solutions to improve the customer experience and the daily internal procedures of the Dublin airport. %I joined the Runway Wall-E team to create a prototype that can identify and locate objects in high contrast with the background. The final objective %was the automation of the landing runway inspection procedure, in order to reduce its timing. The prototype was based on Raspberry Pi 3, while for %programming we used Python for its simplicity and the presence of the OpenCV library}
	%	{}
	%	{}
\end{entrylist}

%----------------------------------------------------------------------------------------
%	ADDITIONAL INFORMATION
%----------------------------------------------------------------------------------------

\begin{minipage}[t]{0.3\textwidth}
	\vspace{-\baselineskip} % Required for vertically aligning minipages

	\cvsect{Languages}
	
	\textbf{Italian} - native\\
	\textbf{English} - proficient\\
	\textbf{Spanish} - rudimentary
\end{minipage}
\hfill
\begin{minipage}[t]{0.3\textwidth}
	\vspace{-\baselineskip} % Required for vertically aligning minipages
	
	\cvsect{Hobbies}
	
	I like to travel and visit cities of art. I am passionate about powerlifting, cinema, motor races, comics and chess
\end{minipage}
\hfill
\begin{minipage}[t]{0.3\textwidth}
	\vspace{-\baselineskip} % Required for vertically aligning minipages
	
	\cvsect{Non profit}
	
	AVIS (Italian Volunteer Blood Donation Association) Pavia member since 2016. Runner for AISM (Italian Multiple Sclerosis Association)
\end{minipage}

%----------------------------------------------------------------------------------------

\end{document}
